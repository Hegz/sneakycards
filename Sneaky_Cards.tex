%Gurps 4th Lite Combat Cards
\documentclass[parskip,letterpaper]{scrartcl}
\usepackage[top=7mm,bottom=7mm,left=7mm,right=7mm]{geometry}


\usepackage[table,usenames,dvipsnames]{xcolor}
\usepackage{tikz}
\usepackage{pifont}
\usepackage{graphicx}
\usepackage{setspace}
\usepackage{dingbat}
\usepackage{fourier}
\usepackage[clock]{ifsym}
\usepackage{dictsym}
\usepackage{wasysym}
\usepackage{multirow}

\pgfmathsetmacro{\cardroundingradius}{4mm}
\pgfmathsetmacro{\striproundingradius}{3mm}
\pgfmathsetmacro{\cardwidth}{6.3}
\pgfmathsetmacro{\cardheight}{8.8}
\pgfmathsetmacro{\stripwidth}{1.0}
\pgfmathsetmacro{\strippadding}{0.1}
\pgfmathsetmacro{\textpadding}{0.3}
\pgfmathsetmacro{\footeroffset}{7.8}
\newcommand{\stripcolor}{cyan}
\newcommand{\striptext}{Sneaky Cards}
\newcommand{\topcaption}{Your Objective}
\newcommand{\topcontent}{}
\newcommand{\bottomcaption}{\bfseries -- Prototype Card -- \\ http://sneakycards.net}
\newcommand{\bottomcontent}{}
\newcommand{\creator}{Harry Lee}

\pgfmathsetmacro{\ruleheight}{0.08}
\newcommand{\stripfontsize}{\Huge}
\newcommand{\captionfontsize}{\Large}
\newcommand{\textfontsize}{\large}
\newcommand{\seperatorline}{\tikz{\fill (0,0) rectangle (\cardwidth-\strippadding-\stripwidth-2*\textpadding,\ruleheight);}\\}

\newcommand{\printcard}{\begin{tikzpicture}
    \draw[rounded corners=\cardroundingradius] (0,0) rectangle (\cardwidth,\cardheight);
    \fill[\stripcolor,rounded corners=\striproundingradius] (\strippadding,\strippadding) rectangle (\strippadding+\stripwidth,\cardheight-\strippadding) node[rotate=90,above left,black,font=\stripfontsize] {\striptext};
    \node[rotate=90,above right,black,font=\scriptsize] at (\strippadding+\stripwidth+0.4,\strippadding) {This card is based on the work of \creator.};
    \node[text width=(\cardwidth-\strippadding-\stripwidth-2*\textpadding)*1cm,below right,inner sep=0] at (\strippadding+\stripwidth+\textpadding,\cardheight-\textpadding) 
    {{\captionfontsize \topcaption \\\vspace{3mm}} 
        {\textfontsize \topcontent \\}
        \seperatorline 
        {\textfontsize \bottomcontent \\}
    };
    \node[text width=(\cardwidth-\strippadding-\stripwidth-2*\textpadding)*1cm,below right,inner sep=0] at (\strippadding+\stripwidth+\textpadding,\cardheight-\textpadding-\footeroffset) 
		{\setstretch{0.75} \scriptsize \centering \bottomcaption\\};
\end{tikzpicture}
}
%\setstretch{0.75}
\begin{document}
\input{en_us/"apprentice wordsmith.tex"}
\renewcommand{\stripcolor}{violet}
	\renewcommand{\striptext}{Comedian}
	\renewcommand{\topcontent}{Make somebody Laugh}
	\renewcommand{\bottomcontent}{When you do: give them this card.}
	\renewcommand{\creator}{Harry Lee}
	\printcard

\input{en_us/"base intrusion.tex"}
\input{en_us/"bear hug.tex"}
\input{en_us/"be brave.tex"}
\input{en_us/"caffine fiend.tex"}
\input{en_us/"chips for free.tex"}
\input{en_us/"collateral damage.tex"}
\input{en_us/"composite sketch.tex"}
\renewcommand{\stripcolor}{red}%
\renewcommand{\striptext}{Doppelg\"anger}%
\renewcommand{\topcontent}{Find this card's exact double}%
\renewcommand{\bottomcontent}{When you do: Hide one card, and give the other away.\\\vspace{10mm}{\setstretch{0.75} \centering \Circle\Circle\Circle\Circle\Circle\Circle\Circle \\\Circle\Circle\Circle\Circle\Circle\Circle\Circle \\\Circle\Circle\Circle\Circle\Circle\Circle\Circle \\\Circle\Circle\Circle\Circle\Circle\Circle\Circle \\\Circle\Circle\Circle\Circle\Circle\Circle\Circle \\\Circle\Circle\Circle\Circle\Circle\Circle\Circle \\\Circle\Circle\Circle\Circle\Circle\Circle\Circle \\\hfill}}%
\renewcommand{\creator}{Harry Lee}%
\printcard%%

\renewcommand{\stripcolor}{red}%
\renewcommand{\striptext}{Doppelg\"anger}%
\renewcommand{\topcontent}{Find this card's exact double}%
\renewcommand{\bottomcontent}{When you do: Hide one card, and give the other away.\\\vspace{10mm}{\setstretch{0.75} \centering \Circle\Circle\Circle\Circle\Circle\Circle\Circle \\\Circle\Circle\Circle\Circle\Circle\Circle\Circle \\\Circle\Circle\Circle\Circle\Circle\Circle\Circle \\\Circle\Circle\Circle\Circle\Circle\Circle\Circle \\\Circle\Circle\Circle\Circle\Circle\Circle\Circle \\\Circle\Circle\Circle\Circle\Circle\Circle\Circle \\\Circle\Circle\Circle\Circle\Circle\Circle\Circle \\\hfill}}%
\renewcommand{\creator}{Harry Lee}%
\printcard%%

\input{en_us/"eye spy.tex"}
\input{en_us/"finders keepers.tex"}
\input{en_us/"finding the beginning.tex"}
\input{en_us/"free beer.tex"}
\renewcommand{\stripcolor}{red}%
\renewcommand{\striptext}{Geocacher}%
\renewcommand{\topcontent}{Leave this card in a Geocache}%
\renewcommand{\bottomcontent}{Go to www.Geocaching.com to find caches near you.}%
\renewcommand{\creator}{SirKibble}%
\printcard%%

\input{en_us/"haiku for you.tex"}
\renewcommand{\stripcolor}{red}%
\renewcommand{\striptext}{Investigator}%
\renewcommand{\topcontent}{Find the Putpocket card}%
\renewcommand{\bottomcontent}{Follow the trail.  When you find it, claim the card and leave this one in its place.}%
\renewcommand{\creator}{Hegzdesimal}%
\printcard%%

\input{en_us/"kick me.tex"}
\input{en_us/"lost and found.tex"}
\renewcommand{\stripcolor}{yellow}%
\renewcommand{\striptext}{Masquerade}%
\renewcommand{\topcontent}{Impersonate somebody famous}%
\renewcommand{\bottomcontent}{Your objective is complete when somebody guesses your identity.  When they do, give them this card.}%
\renewcommand{\creator}{Hegzdesimal}%
\printcard%%

\input{en_us/"not it.tex"}
\input{en_us/"once upon a time.tex"}
\input{en_us/"patch painter.tex"}
\renewcommand{\stripcolor}{cyan}%
\renewcommand{\striptext}{Pickpocket}%
\renewcommand{\topcontent}{Take a sneaky card from somebody without them knowing, and leave this card in it's place}%
\renewcommand{\bottomcontent}{If your target catches you stealing their card, give this card away.}%
\renewcommand{\creator}{Hegzdesimal}%
\printcard%%

\input{en_us/"picture this.tex"}
\input{en_us/"plain sight.tex"}
\renewcommand{\stripcolor}{yellow}%
\renewcommand{\striptext}{Putpocket}%
\renewcommand{\topcontent}{Give this card to somebody without them knowing}%
\renewcommand{\bottomcontent}{if your target catches you giving this card to them, you must take back this card.}%
\renewcommand{\creator}{Harry Lee}%
\printcard%%
  
\input{en_us/"reverse hustle.tex"}
\input{en_us/"rubbing elbows.tex"}
\renewcommand{\stripcolor}{cyan}
	\renewcommand{\striptext}{Showdown}
	\renewcommand{\topcontent}{Defeat somebody in a duel}
	\renewcommand{\bottomcontent}{Challenge somebody to a contest (eg, Arm Wrestle).\\\vspace{2mm} If you are victorious, give this card to that person}
	\renewcommand{\creator}{Harry Lee}
	\printcard

\input{en_us/"secrets of the library.tex"}
\input{en_us/"serious business.tex"}
\input{en_us/"sharing peace.tex"}
\input{en_us/"sleeping beauty.tex"}
\input{en_us/"star struck.tex"}
\input{en_us/"thank you.tex"}
\input{en_us/"the come on.tex"}
\input{en_us/"the conversationist.tex"}
\input{en_us/"the gift of giving.tex"}
\input{en_us/"the hobbyist.tex"}
\input{en_us/"the investor.tex"}
\input{en_us/"the long con.tex"}
\input{en_us/"the meta card.tex"}
\input{en_us/"the meow card.tex"}
\input{en_us/"the passenger.tex"}
\input{en_us/"the riddler.tex"}
\renewcommand{\stripcolor}{cyan}
	\renewcommand{\striptext}{Tribute}
	\renewcommand{\topcontent}{Sincerely complement somebody}
	\renewcommand{\bottomcontent}{When you do: give them this card.}
	\printcard

\input{en_us/"wallstreet.tex"}
\input{en_us/"yanky doodle.tex"}
\input{en_us/"very merry unbirthday.tex"}
\input{en_us/"you're it.tex"}
\end{document}
%BLUE cards test your audacity and chutzpah.
%YELLOW cards require sneakiness and espionage skills.
%RED cards involve finding things - and not just objects.
%GREEN cards are about goodwill and giving to others.
%PURPLE cards will plague your brain with puzzles.
%ORANGE cards challenge you to create art with purpose.
%SILVER cards are information cards.
% -- Defeat somebody in a duel.
%Your objective: Get revenge. Give the revengee this card.
%Your objective: reverse hustle someone. Forcibly give them money, then give them this card.
% -- Give this card to somebody without them knowing.
%Your objective: Get this card signed by a celebrity, then pass it on to someone else.
%Make someone lose the game. Place this card as a trap for somebody else.
% -- Find this card's exact double.
% -- Find the object circled on the list. When you have found it, cross it off and circle a new object, then give this card away.
% -- Place this card someone highly visible but hard to reach. Your objective is complete when someone has removed the card
% -- Give a gift to somebody, along with this card.
% -- Make somebody laugh.
%Return a lost item to it's owner, along with this card.
%Find someone who needs cheering up. Give them this card, and something to cheer them up.
% -- Add one word to the story on the back, then give this card to someone else.
